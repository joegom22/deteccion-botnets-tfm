\chapter{Estado de la Cuestión}
\label{cap:estadoDeLaCuestion}

En esta sección se presentará el estado actual de la detección de \textit{botnets} en redes IoT. Para ello, se comienza introduciendo dos conceptos fundamentales, por un lado el de Redes de Bots y, por otro el de Aprendizaje Automático. 


Una vez habiendo sentado estas bases se procederá a analizar la situación actual del campo de la detección de este tipo de redes maliciosas empleando técnicas de Aprendizaje Automático.

\section{Conceptos Fundamentales}
Para una correcta comprensión del presente trabajo es crucial que el lector esté familiarizado tanto con las \textit{botnets} como con el aprendizaje automático.
\subsection{Botnets: Una Amenaza en Auge}
Considérese una red de dispositivos IoT. Se denomina \textit{botnet} a una subred de dichos dispositivos que se encuentran controlados remotamente por un usuario externo denominado \textit{"Botmaster"}.


Estos aparatos, al pertenecer a la red IoT original, son empleados por su nuevo dueño para realizar ataques o acciones maliciosas en perjuicio de la red y/o en beneficio del \textit{"Botmaster"}. Entre los ataques más habituales pueden encontrarse:
\begin{itemize}
	\item \textbf{Ataque de Denegación de Servicio (DDOS):} En un ataque de este tipo los dispositivos "infectados" buscan imposibilitar a la víctima su uso del servicio (por ejemplo, su conexión a Internet). Para ello se dedican a enviar una gran cantidad de paquetes inutiles a través de la red buscando congestionarla hasta que no pueda abarcar la gestión de tantos mensajes y deje de poder funcionar correctamente. \\
	\textbf{Ejemplo IoT:} Una red de dispositivos IoT de monitorización de cabezas de ganado por ejemplo podría ser víctima de un ataque de este tipo. En este caso, el \textit{"Botmaster"} podría inundar el servidor que controla las reses con mensajes de posición hasta que colapsara. De esta forma este no sería capaz de detectar, por ejemplo, la desaparición o rapto de un animal.
	\item \textbf{Robo de Información/Espionaje:} En este caso los dispositivos controlados formarían parte de alguna red de recolección de información (por ejemplo sensores o cámaras en un hogar inteligente) y el ataque consistiría en desviar la información recogida para poder, por ejemplo, chantajear al dueño original.
	\item \textbf{Minado de Criptomonedas (Cryptojacking):} Este tipo de ataque, de gran interés desde el auge de Bitcoin en la última década, consiste en emplear los dispositivos de la víctima para minar cryptomonedas sin su conocimiento. 
\end{itemize}
Las redes IoT, compuestas normalmente por muy diversos y muy alejados dispositivos, son un objetivo especialmente vulnerable ante este tipo de ataques.
\subsection{Mirai: Crisis de Denegación de Servicio}
Es a finales de 2016 cuando comienzan a sucederse ataques de Denegación de Servicio que inutilizaron algunas grandes compañías tecnológicas como la francesa OVH o la estadounidense DynDNS. 


 Creada en sus inicios por un grupo de 3 jóvenes estadounidenses la Red de Bots Mirai fue utilizada por diversos hackers después de que su código fuente fuera publicado en varios foros de Internet. 
 
 
 Mirai buscaba realizar un Ataque de Denegación de Servicio y, para ello, se valía de Ataques de Fuerza Bruta para adivinar claves de dispositivos IoT que, muchas veces, eran aquellas que el fabricante creó por defecto con poco grado de complejidad. Una vez se había logrado apoderar del dispositivo usaba su nueva red de bots para saturar los servidores mediante la generación masiva de tráfico. 
 
 
 Entre los tipos de ataques DDOS que se realizaban destacan:
 \begin{itemize}
 	\item \textbf{SYN flood:} Este ataque consiste en iniciar rápidamente conexiones al servidor sin llegar a finalizarlas, obligando a este a destinar recursos en esperar a estas conexiones parciales. Impidiendo así el tráfico legítimo.
 	\item \textbf{ACK flood:} En este caso se activa el \textit{flag} ACK en una gran cantidad de \textit{requests} de forma que el envío de las mismas y su recepción colapsa el servidor.
 	\item \textbf{UDP flood:} En este ataque el servidor recibirá una gran cantidad de paquetes UDP a puertos aleatorios, lo cual obligará al servidor a gestionarlos y, generalmente, responder con paquetes ICMP informando de que no hay ninguna aplicación escuchando en dicho puerto. Esta gran cantidad de paquetes desborda la capacidad del servidor.
 	\item \textbf{HTTP flood:} En un ataque HTTP flood los \textit{bots} envían solicitudes HTTP (generalmente GET o POST) para lograr que el servidor tenga que dedicar sus recursos a gestionar dicho ataque.
\end{itemize}
El ataque de Mirai puso de manifiesto la vulnerabilidad de las redes IoT y los peligros que este nuevo paradigma traía consigo si se descuidaba la seguridad.
\subsection{Aprendizaje Automático: Detectando lo Indetectable}
