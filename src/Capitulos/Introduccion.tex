\chapter{Introducción}
\label{cap:introduccion}

\chapterquote{Frase célebre dicha por alguien inteligente}{Autor}


En los últimos años, los avances en el ámbito del Internet de las Cosas y la rápida proliferación de dispositivos IoT han traído consigo importantes mejoras en la conectividad, permitiendo la automatización y el monitoreo en contextos que van desde el hogar hasta la industria.

Sin embargo, la propia idiosincrasia del Internet de las Cosas, en su heterogeneidad, lo hace un objetivo ideal para los ataques con redes de bots (\textit{Botnets}). Es en este contexto donde aparece el aprendizaje automático como herramienta para la detección de estas amenazas.

En este trabajo se estudiarán e implementarán diferentes modelos de aprendizaje automático para comprobar cuáles de ellos son los más apropiados para detectar los distintos tipos de posibles \textit{botnets}.

\section{Motivación}


\textit{Mirai}, \textit{Perisai} o \textit{Matrix} son solo algunas de las \textit{botnets} que han protagonizado recientemente noticias por su infiltración y ataque en redes IoT. 

Los avances tecnológicos de la última década no solo mejoran las vidas de los ciudadanos, también permiten que estos ataques sean cada vez más complejos. Los métodos tradicionales para hacerles frente quedan obsoletos y se presenta la necesidad de desarrollar sistemas más inteligentes para poder hacer frente a esta amenaza, el aprendizaje automático.

\section{Objetivos}
El presente trabajo tiene como objetivo principal diseñar y evaluar diferentes sistemas de detección de \textit{botnets} en redes IoT haciendo uso de técnicas de aprendizaje automático.

Además, se busca discutir las ventajas y limitaciones de los enfoques propuestos para poder determinar los casos de uso más apropiados de cada uno de ellos. Como último objetivo, se quiere presentar un resumen de algunas de las técnicas de implementación de contramedidas frente a las \textit{botnets} detectadas.

\section{Plan de trabajo}
Para conseguir los objetivos anteriormente mencionados se empleará el siguiente proceso de trabajo.

